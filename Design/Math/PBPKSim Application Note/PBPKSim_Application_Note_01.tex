\documentclass[12pt]{scrartcl}

\usepackage[cmex10]{amsmath}
\usepackage{amssymb}
\usepackage{color}

\newcommand{\phicl}{\varphi_{\mathrm{cl}}}                                                          
\newcommand{\N}{\mathbb{N}} 
\newcommand{\x}{\mathbf{x}} 
\newcommand{\conv}{\operatorname{co}}
\newcommand{\cone}{\operatorname{cone}}
\newcommand{\Pre}{\operatorname{Pre}}  
\newcommand{\dom}{\operatorname{dom}} 
\newcommand{\proj}{\operatorname{proj}} 
\newcommand{\diag}{\operatorname{diag}} 
\newcommand{\cl}{\operatorname{cl}}
\newcommand{\interior}{\operatorname{int}}
\newcommand{\dist}{\operatorname{dist}}
\newcommand{\X}{{X}}
\newcommand{\Us}{{U}}
\newcommand{\Ss}{\mathcal{S}}
\newcommand{\Y}{{Y}}
\newcommand{\Z}{{Z}}
\renewcommand{\Re}{\mathbb{R}}
\newcommand{\dfn}{\mathrel{\mathop:}=} 
\newcommand{\seq}[2]{\left\langle#1\right\rangle_{#2\in\mathbb{N}}}
\newcommand{\smallmat}[1]{
  \left[ \begin{smallmatrix}#1 \end{smallmatrix} \right]}
\newcommand{\ie}{\textit{i.e.}}

\newcommand{\bcj}{\bar{c}_j}
\newcommand{\buj}{\bar{u}_j}


\title{Formulation of the MPC problem as a QP problem}
\subtitle{PBPKSim Application Note}
\author{Pantelis~Sopasakis}
        
\begin{document}
\maketitle

\section*{Scope}
In this short application note we present how 
one can formulate the offset-free MPC control problem
as a quadratic optimisation problem which can be solved
with any solver in MATLAB, Python or other languages. 

\section*{Formulation of the problem}

\subsection*{Notation}
Let $\N$, $\Re$, $\Re^n$, $\Re^{n\times m}$ denote the set
of non-negative integers, the set of real number, the set of $n$-vectors,
and the set of $n$-by-$m$ matrices. Let $\N_{[k_1, k_2]}$ denote the
set of non-negative integers that are between $k_1$ and $k_2$.
We denote by $I_m$ the unit matrix of size $m$-by-$m$; the simpler
notation $I$ will sometimes be used without explicitly stating the
dimension of the matrix.

\subsection*{General formulation}
The cost function is defined as:
\begin{align}\label{eq:cost-definition}
V_N(\pi, \hat{c}_j, \hat{d}_j) = \|c_N - \bar{c}_j\|_P^2 + \sum_{k=0}^{N-1}\left( 
\|c_k-\bar{c}_j\|_Q^2+\|u_k-\bar{u}_j\|_R^2
\right),
\end{align}
where $\|\cdot\|_Q$ is the $Q$-(semi)norm defined as $\|c\|_Q^2=c'Qc$ for 
a positive (semi)definite matrix $Q$, $c_k\in\Re^n$ is the (predicted) state of the system,
$u_k\in\Re^m$ is a (predicted) input, $\bar{c}_j$ and $\bar{u}_j$ are reference values
computed as explained in~\cite[Eq. (33)]{SopPatSar14} and $P$ is a 
properly calculated terminal cost matrix~\cite[Eq. (32)]{SopPatSar14}.
In~\eqref{eq:cost-definition}, $\pi$ stands for a sequence of inputs
$\pi=\{u_i\}_{i=0}^{N-1}$ of length $N$.

We assume that a constant additive disturbance $d_k\in\Re^{n_d}$ acts on the system,
where $n_d$ is less than of equal to the number of measured outputs of the system.
The system dynamics is described by:
\begin{align}
V_N^\star(\hat{c}_j, \hat{d}_j) = \min_{\pi=\{u_i\}_{i=0}^{N-1}}V_N(\pi, \hat{c}_j, \hat{d}_j),
\end{align}
subject to the constraints:
\begin{subequations}
\begin{align}
c_{k+1} = &A c_k + Bu_k + B_d d_k, \forall k\in\N_{[0,N-1]},\label{eq:state-update}\\
d_{k+1} &= d_k, \forall k\in\N_{[0,N-1]},\label{eq:disturbance-dynamics}\\
Gc_k + Hu_k &\leq M, \forall k\in\N_{[0,N-1]},\label{eq:mixed-state-input-constraints}\\
G_fc_N &\leq M_f,\label{eq:mixed-state-input-constraints-terminal}\\
c_0 &= \hat{c}_j,\\
d_0 &= \hat{d}_j,
\end{align}

but where the mixed state-input constraints of~\eqref{eq:mixed-state-input-constraints}
and the terminal constraints~\eqref{eq:mixed-state-input-constraints-terminal}
can sometimes be simplified and written in the form:
\begin{align}
0 \leq &c_k \leq c_{\max}, \forall  k\in\N_{[1,N]}\label{eq:state-bounds},\\
0 \leq &u_k \leq u_{\max}, \forall  k\in\N_{[0,N-1]}\label{eq:input-bounds}.
\end{align}
\end{subequations}
Let us define the optimisation variable $z$ as follows:
\begin{align}
z=\left( u_0, c_1, u_1, c_2, \ldots, c_{N-1}, u_{N-1}, c_{N}\right)',
\end{align}
where $z\in\Re^{N(n+m)}$. Then, the cost function in~\eqref{eq:cost-definition}
can be written as follows:
\begin{align*}
V_N &{=} (c_N-\bcj)'P(c_N-\bcj) {+} \sum_{k=0}^{N-1}(c_k-\bcj)'Q(c_k-\bcj){+} 
		(u_k-\buj)'R(u_k-\buj)\\
		&= c_N' P c_N + \bcj' P \bcj - \bcj' P c_N - c_N' P \bcj +\\
		  &\qquad + \sum_{k=0}^{N-1} c_k' Q c_k + \bcj' Q c_k + u_k' R u_k + \buj' R \buj - 2\buj R u_k\\
		&=\textcolor{red}{c_N' P c_N + \sum_{k=0}^{N-1} \left[ \begin{array}{cc}c_k' & u_k'\end{array}\right]
\left[ \begin{array}{cc}
Q & 0\\0 & R
\end{array}\right]
\left[ \begin{array}{c}
c_k\\
u_k
\end{array}\right]}+\\
&\qquad+\textcolor{blue}{\left( 
\sum_{k=0}^{N-1}\left[ \begin{array}{cc}
-2\bcj Q & -2\buj R
\end{array}\right]\left[ 
\begin{array}{c}
c_k\\u_k
\end{array}\right]
\right)}+\textcolor{magenta}{\gamma},
\end{align*}
where the terms in red are quadratic terms, the blue ones are linear and
there is also a constant term $\gamma$ which we shall omit.

The cost function $V_N$ can be now written as a function of $z$ in the following
form:
\begin{align}
V_N(z) = \frac{1}{2}z' \Theta z + q'z,
\end{align}
where $\Theta\in\Re^{N(n+m)\times N(n+m)}$ is the following square positive definite matrix:
\begin{align}
\Theta=2\left[
\begin{array}{cccccccc}
R\\
&Q\\
&&R\\
&&&Q\\
&&&&\ddots\\
&&&&&Q\\
&&&&&&R\\
&&&&&&&P
\end{array}\right],
\end{align}
and $q\in \Re^{N(n+m)}$ is:
\begin{align}
q=-2\left[
\begin{array}{c}
\buj' R\\
\bcj' Q\\
\vdots\\
\buj' R\\
\bcj' Q\\
\bcj' P
\end{array}\right],
\end{align}
If we only have constraints of the simple form~\eqref{eq:state-bounds} and~\eqref{eq:input-bounds},
then, the constraints on $z$ are also bounds of the form $0\leq z \leq z_{\max}$, where
$z_{\max}\in\Re^{N(n+m)}$ is the vector:
\begin{align}
z_{\max}=\left[
\begin{array}{c}
u_{\max}\\
c_{\max}\\
\vdots\\
u_{\max}\\
c_{\max}\\
\end{array}\right].
\end{align}
Given that the disturbance dynamics is~\eqref{eq:disturbance-dynamics}, the
state-update equation~\eqref{eq:state-update} can be written as follows:
\begin{align*}
c_1 - Bu_0 &= B_d d_0 + Ac_0,\\
c_2 - Ac_1 - Bu_1 &= B_d d_0,\\
c_3 - Ac_2 - Bu_2 &= B_d d_0,\\
&\ \ \vdots\\
c_N - Ac_{N-1}-Bu_{N-1} &= B_d d_0,
\end{align*}
which is compactly written in the matrix form $Kz = L$ as follows:
\begin{align}
K=\left[\begin{array}{ccccccc}
-B & I\\
& -A & -B &I\\
&& -A & -B &I\\
&&&\ddots & \ddots & \ddots\\
&&&&-A&-B&I
\end{array}\right],
\end{align}
and
\begin{align}
L=\left[
\begin{array}{c}
B_d d_0 + Ac_0\\
B_d d_0\\
B_d d_0\\
\vdots\\
B_d d_0
\end{array}\right],
\end{align}
where, of course $d_0=\hat{d}_j$ and $c_0=\hat{c}_j$, that is
$d_0$ is the current disturbance estimate and $c_0$ is the current
concentration (state) estimate.

Now, overall the problem has been formulated as a convex QP as follows:
\begin{subequations}
\begin{align}
V_N^\star = \min_{z} \frac{1}{2}z'\Theta z + q'z,
\end{align}
subject to the constraints:
\begin{align}
0 &\leq z \leq z_{\max}\label{eq:z-bound}\\
Kz &= L.
\end{align}
\end{subequations}

\subsection*{Mixed state-input constraints}
There are cases where we need to impose mixed state-input constraints 
as in~\eqref{eq:mixed-state-input-constraints} and/or
terminal constraints as in~\eqref{eq:mixed-state-input-constraints-terminal}.
These constraints may be imposed additionally to the bounds~\eqref{eq:z-bound}.
Constraints of the form~\eqref{eq:mixed-state-input-constraints} can be 
represented, in terms of $z$, in the form:
\begin{align}
\Phi z \leq \phi,
\end{align}
where
\begin{equation}
\Phi=\left[\begin{array}{cccccc}
H & 0\\
& G  & H\\
&& G & H \\
&&& \ddots & \ddots\\
&&&& G & H
\end{array}\right],
\end{equation}
and
\begin{align}
\phi=\left[ \begin{array}{c}
M-Gc_0\\
M\\
\vdots\\
M
\end{array}\right].
\end{align}





\subsection*{Partial constraints}
Constraints may not be necessarily imposed on all concentrations but
only on some $c^j$ for $j\in J$, where $c^j$ stands for the $j$-th element of the 
vector $c\in\Re^n$. Let $J=\{j_1,j_2,\ldots, j_l\}$ with $l\leq n$. Let
\begin{equation}
\Delta_J = \left[\begin{array}{c}
e(j_1)'\\
e(j_2)'\\
\vdots\\
e(j_k)'
\end{array}\right],
\end{equation}
where $e(j)$ is a vector of $\Re^n$ whose $j$-th entry
is $1$ and all other entries are equal to $0$. The constraints
on $c$ are then written as:
\begin{subequations}
\begin{align}
\Delta_J c_k &\leq c^J_{\max}, \forall k\in\N_{[0,N]},\\
0 &\leq c_k, \forall k\in\N_{[0,N]},
\end{align}
\end{subequations}
where $c^J_{\max}\in\Re^{l}$ stands for the bounds on $c^{j}$ with $j\in J$. 
In terms of the variable $z$, the state and input bounds~\eqref{eq:state-bounds} and~\eqref{eq:input-bounds}
can be written as $\Xi_J z \leq \xi_J$, where
\begin{subequations}
\begin{align}
\Xi_J=\left[\begin{array}{ccccc}
I_m\\
&\Delta_J\\
&&\ddots\\
&&&I_m\\
&&&&\Delta_J
\end{array}\right],
\end{align}
and
\begin{align}
\xi_J = \left[\begin{array}{c}
u_{\max}\\
c^J_{\max}\\
u_{\max}\\
c^J_{\max}\\
\vdots\\
u_{\max}\\
c^J_{\max}
\end{array}\right].
\end{align}
\end{subequations}


\bibliographystyle{ieeetr}
\bibliography{appl-note-bib}

\end{document}